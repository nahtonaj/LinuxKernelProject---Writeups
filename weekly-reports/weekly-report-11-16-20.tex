% !TeX root = ./weekly-report-11-16-20.tex
\documentclass[12pt]{article}
% \usepackage{fullpage}
\usepackage{epic}
\usepackage{eepic}
\usepackage{paralist}
\usepackage{graphicx}
\usepackage{algorithm,algorithmic}
\usepackage{tikz}
\usepackage{xcolor,colortbl}
\usepackage{wrapfig}
\usepackage{float}


%%%%%%%%%%%%%%%%%%%%%%%%%%%%%%%%%%%%%%%%%%%%%%%%%%%%%%%%%%%%%%%%
% This is FULLPAGE.STY by H.Partl, Version 2 as of 15 Dec 1988.
% Document Style Option to fill the paper just like Plain TeX.

\typeout{Style Option FULLPAGE Version 2 as of 15 Dec 1988}

\topmargin 0pt
\advance \topmargin by -\headheight
\advance \topmargin by -\headsep

\textheight 8.9in

\oddsidemargin 0pt
\evensidemargin \oddsidemargin
\marginparwidth 0.5in

\textwidth 6.5in
%%%%%%%%%%%%%%%%%%%%%%%%%%%%%%%%%%%%%%%%%%%%%%%%%%%%%%%%%%%%%%%%

\pagestyle{empty}
\setlength{\oddsidemargin}{0in}
\setlength{\topmargin}{-0.8in}
\setlength{\textwidth}{6.8in}
\setlength{\textheight}{9.5in}


\def\ind{\hspace*{0.3in}}
\def\gap{0.1in}
\def\bigap{0.25in}
\newcommand{\Xomit}[1]{}


\begin{document}

\setlength{\parindent}{0in}
\addtolength{\parskip}{0.1cm}
\setlength{\fboxrule}{.5mm}\setlength{\fboxsep}{1.2mm}
\newlength{\boxlength}\setlength{\boxlength}{\textwidth}
\addtolength{\boxlength}{-4mm}
\begin{center}\framebox{\parbox{\boxlength}{{\bf
MENG PROJECT, FA20\hfill Linux Kernel Modifications}\\
% TODO: fill in your own name, netID, and collaborators
Name: Jonathan Gao\hfill 
NetID: jg992
}}
\end{center}
\vspace{2mm}

\section*{Progress this week}
\ind Dug around in the kernel a little bit and found the implementation of the CFS scheduler. There is some method \texttt{pick\_next\_task()} that seems to branch out into the rest of the different scheduler implementations. I still need to really dig into how exactly this function actually goes through the queues and picks the tasks. It seems like there is just a simple iterator through the different scheduling runqueues that finds the first task. But, it's all in these weird wrapper functions with weird annotations that I haven't seen before. But, if this function is really all there is to picking a task, it might not be that bad to implement something like a MLFQ using the SCHED\_RR queues.

PREEMPT\_RT patch did not really yield anything, if anything it's even harder to parse through since its just diffs, not anything cohesive. It doesn't add new schedulers and rather just patches the old one I think.

\section*{Plan for coming week}
As above, to dig into how exactly the function goes through the queues and picks the tasks. I might be able to just pop in my scheduler into this function.

\section*{Problems}

\section*{Long term plan: still OK?}
Hopefully so! 





\end{document}

